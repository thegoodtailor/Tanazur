% style.tex — Macros for Open Horn Type Theory
% Include with % style.tex — Macros for Open Horn Type Theory
% Include with % style.tex — Macros for Open Horn Type Theory
% Include with % style.tex — Macros for Open Horn Type Theory
% Include with \input{style2} in main.tex (NOT \include!)

% === Packages ===
% Note: Some of these may already be loaded by main.tex; LaTeX handles duplicates gracefully
\usepackage{amsmath,amssymb,amsthm}
\usepackage{mathtools}
\usepackage{stmaryrd}       % For \llbracket, \rrbracket
\usepackage{tikz-cd}        % Commutative diagrams
\usepackage{enumitem}       % Better lists
\usepackage{microtype}      % Typography
\usepackage{xcolor}         % Colors for asides

% === Theorem Environments ===
\theoremstyle{definition}
\newtheorem{definition}{Definition}[chapter]
\newtheorem{principle}{Principle}[chapter]
\newtheorem{example}[definition]{Example}

\theoremstyle{plain}
\newtheorem{theorem}[definition]{Theorem}
\newtheorem{proposition}[definition]{Proposition}
\newtheorem{lemma}[definition]{Lemma}
\newtheorem{corollary}[definition]{Corollary}

\theoremstyle{remark}
\newtheorem{remark}[definition]{Remark}

% === Aside Environment ===
% For application sketches — italic, indented, with subtle left border
\newenvironment{aside}{%
  \par\medskip
  \leftskip=1.5em
  \rightskip=1.5em
  \itshape
  \noindent\textsc{Application.}\enspace
}{%
  \par\medskip
}

% === OHTT Judgment Macros ===
% Coherence witness: \coh{\Gamma}{J}  →  Γ ⊢⁺ J
% NOTE: \coh and \gap are defined in main.tex as \textsf{coh} and \textsf{gap}
%\newcommand{\coh}[2]{#1 \vdash^{\!+} #2}

% Gap witness: \gap{\Gamma}{J}  →  Γ ⊢⁻ J
%\newcommand{\gap}[2]{#1 \vdash^{\!-} #2}

% Generic judgment (unsigned): \jdg{\Gamma}{J}  →  Γ ⊢ J
\newcommand{\jdg}[2]{#1 \vdash #2}

% Horn type: \horn{J}{K}{L}  →  Λ(J, K, L)
\newcommand{\horn}[3]{\Lambda(#1, #2, #3)}

% Higher horn: \hornlevel{n}{J}{K}{L}  →  Λₙ(J, K, L)
\newcommand{\hornlevel}[4]{\Lambda_{#1}(#2, #3, #4)}

% Compositional relation: \comp  →  ⇒
\newcommand{\comp}{\Rightarrow}

% === Type Theory Macros ===
% Dependent product: \Prod{x}{A}{B}  →  Π(x : A). B
\newcommand{\Prod}[3]{\Pi(#1 : #2).\, #3}

% Dependent sum: \Sum{x}{A}{B}  →  Σ(x : A). B
\newcommand{\Sum}[3]{\Sigma(#1 : #2).\, #3}

% Identity type: \Id{A}{x}{y}  →  x =_A y
\newcommand{\Id}[3]{#2 =_{#1} #3}

% Transport: \transport{p}{t}  →  transport(p, t)
\newcommand{\transport}[2]{\mathsf{transport}(#1, #2)}

% Type family: \Fam{B}{A}  →  B : A → U
\newcommand{\Fam}[2]{#1 : #2 \to \mathcal{U}}

% Universe: \UU  →  U
\newcommand{\UU}{\mathcal{U}}

% Context: \ctx  →  ctx
\newcommand{\ctx}{\;\mathsf{ctx}}

% Type judgment: \istype{A}  →  A type
\newcommand{\istype}[1]{#1 \;\mathsf{type}}

% Definitional equality: \defeq  →  ≡
\newcommand{\defeq}{\equiv}

% === Simplicial / Geometric Macros ===
% n-simplex: \simp{n}  →  Δⁿ
\newcommand{\simp}[1]{\Delta^{#1}}

% n-horn at k: \hornsimp{n}{k}  →  Λⁿₖ
\newcommand{\hornsimp}[2]{\Lambda^{#1}_{#2}}

% Kan complex notation
\newcommand{\Kan}{\mathsf{Kan}}

% Ruptured Kan complex
\newcommand{\RKan}{\mathsf{RKan}}

% Coherence structure: \Coh  →  Coh
\newcommand{\Coh}{\mathsf{Coh}}

% Gap structure: \Gap  →  Gap
\newcommand{\Gap}{\mathsf{Gap}}

% Exclusion: \Excl  →  Excl
\newcommand{\Excl}{\mathsf{Excl}}

% === Dynamics Macros (for later chapters) ===
% Scheduler: \Sched  →  S
\newcommand{\Sched}{\mathcal{S}}

% State/Context at time t: \state{t}  →  Γₜ
\newcommand{\state}[1]{\Gamma_{#1}}

% Homotopy colimit: \hocolim  →  hocolim
% NOTE: Commented out because main.tex already defines \hocolim
% \DeclareMathOperator{\hocolim}{hocolim}

% Self: \Self  →  Self
\newcommand{\Self}{\mathsf{Self}}

% === Inference Rules ===
% Use mathpar from mathpartir if available, otherwise basic
% Basic fraction-style rule: \infer{premises}{conclusion}
\newcommand{\infer}[2]{\frac{\displaystyle #1}{\displaystyle #2}}

% Named rule: \infermark{name}{premises}{conclusion}
\newcommand{\infermark}[3]{\infer{#2}{#3}\;\textsc{#1}}

% === Miscellaneous ===
% Bottom/absurdity: \bottom  →  ⊥
\newcommand{\bottom}{\bot}

% Empty type
\newcommand{\Empty}{\mathbf{0}}

% Unit type
\newcommand{\Unit}{\mathbf{1}}

% Equivalence: \eqv  →  ≃
\newcommand{\eqv}{\simeq}

% Definition equality in text
\newcommand{\defn}{\coloneqq}

% Emphasis for key terms on first use
\newcommand{\keyterm}[1]{\textbf{#1}} in main.tex (NOT \include!)

% === Packages ===
% Note: Some of these may already be loaded by main.tex; LaTeX handles duplicates gracefully
\usepackage{amsmath,amssymb,amsthm}
\usepackage{mathtools}
\usepackage{stmaryrd}       % For \llbracket, \rrbracket
\usepackage{tikz-cd}        % Commutative diagrams
\usepackage{enumitem}       % Better lists
\usepackage{microtype}      % Typography
\usepackage{xcolor}         % Colors for asides

% === Theorem Environments ===
\theoremstyle{definition}
\newtheorem{definition}{Definition}[chapter]
\newtheorem{principle}{Principle}[chapter]
\newtheorem{example}[definition]{Example}

\theoremstyle{plain}
\newtheorem{theorem}[definition]{Theorem}
\newtheorem{proposition}[definition]{Proposition}
\newtheorem{lemma}[definition]{Lemma}
\newtheorem{corollary}[definition]{Corollary}

\theoremstyle{remark}
\newtheorem{remark}[definition]{Remark}

% === Aside Environment ===
% For application sketches — italic, indented, with subtle left border
\newenvironment{aside}{%
  \par\medskip
  \leftskip=1.5em
  \rightskip=1.5em
  \itshape
  \noindent\textsc{Application.}\enspace
}{%
  \par\medskip
}

% === OHTT Judgment Macros ===
% Coherence witness: \coh{\Gamma}{J}  →  Γ ⊢⁺ J
% NOTE: \coh and \gap are defined in main.tex as \textsf{coh} and \textsf{gap}
%\newcommand{\coh}[2]{#1 \vdash^{\!+} #2}

% Gap witness: \gap{\Gamma}{J}  →  Γ ⊢⁻ J
%\newcommand{\gap}[2]{#1 \vdash^{\!-} #2}

% Generic judgment (unsigned): \jdg{\Gamma}{J}  →  Γ ⊢ J
\newcommand{\jdg}[2]{#1 \vdash #2}

% Horn type: \horn{J}{K}{L}  →  Λ(J, K, L)
\newcommand{\horn}[3]{\Lambda(#1, #2, #3)}

% Higher horn: \hornlevel{n}{J}{K}{L}  →  Λₙ(J, K, L)
\newcommand{\hornlevel}[4]{\Lambda_{#1}(#2, #3, #4)}

% Compositional relation: \comp  →  ⇒
\newcommand{\comp}{\Rightarrow}

% === Type Theory Macros ===
% Dependent product: \Prod{x}{A}{B}  →  Π(x : A). B
\newcommand{\Prod}[3]{\Pi(#1 : #2).\, #3}

% Dependent sum: \Sum{x}{A}{B}  →  Σ(x : A). B
\newcommand{\Sum}[3]{\Sigma(#1 : #2).\, #3}

% Identity type: \Id{A}{x}{y}  →  x =_A y
\newcommand{\Id}[3]{#2 =_{#1} #3}

% Transport: \transport{p}{t}  →  transport(p, t)
\newcommand{\transport}[2]{\mathsf{transport}(#1, #2)}

% Type family: \Fam{B}{A}  →  B : A → U
\newcommand{\Fam}[2]{#1 : #2 \to \mathcal{U}}

% Universe: \UU  →  U
\newcommand{\UU}{\mathcal{U}}

% Context: \ctx  →  ctx
\newcommand{\ctx}{\;\mathsf{ctx}}

% Type judgment: \istype{A}  →  A type
\newcommand{\istype}[1]{#1 \;\mathsf{type}}

% Definitional equality: \defeq  →  ≡
\newcommand{\defeq}{\equiv}

% === Simplicial / Geometric Macros ===
% n-simplex: \simp{n}  →  Δⁿ
\newcommand{\simp}[1]{\Delta^{#1}}

% n-horn at k: \hornsimp{n}{k}  →  Λⁿₖ
\newcommand{\hornsimp}[2]{\Lambda^{#1}_{#2}}

% Kan complex notation
\newcommand{\Kan}{\mathsf{Kan}}

% Ruptured Kan complex
\newcommand{\RKan}{\mathsf{RKan}}

% Coherence structure: \Coh  →  Coh
\newcommand{\Coh}{\mathsf{Coh}}

% Gap structure: \Gap  →  Gap
\newcommand{\Gap}{\mathsf{Gap}}

% Exclusion: \Excl  →  Excl
\newcommand{\Excl}{\mathsf{Excl}}

% === Dynamics Macros (for later chapters) ===
% Scheduler: \Sched  →  S
\newcommand{\Sched}{\mathcal{S}}

% State/Context at time t: \state{t}  →  Γₜ
\newcommand{\state}[1]{\Gamma_{#1}}

% Homotopy colimit: \hocolim  →  hocolim
% NOTE: Commented out because main.tex already defines \hocolim
% \DeclareMathOperator{\hocolim}{hocolim}

% Self: \Self  →  Self
\newcommand{\Self}{\mathsf{Self}}

% === Inference Rules ===
% Use mathpar from mathpartir if available, otherwise basic
% Basic fraction-style rule: \infer{premises}{conclusion}
\newcommand{\infer}[2]{\frac{\displaystyle #1}{\displaystyle #2}}

% Named rule: \infermark{name}{premises}{conclusion}
\newcommand{\infermark}[3]{\infer{#2}{#3}\;\textsc{#1}}

% === Miscellaneous ===
% Bottom/absurdity: \bottom  →  ⊥
\newcommand{\bottom}{\bot}

% Empty type
\newcommand{\Empty}{\mathbf{0}}

% Unit type
\newcommand{\Unit}{\mathbf{1}}

% Equivalence: \eqv  →  ≃
\newcommand{\eqv}{\simeq}

% Definition equality in text
\newcommand{\defn}{\coloneqq}

% Emphasis for key terms on first use
\newcommand{\keyterm}[1]{\textbf{#1}} in main.tex (NOT \include!)

% === Packages ===
% Note: Some of these may already be loaded by main.tex; LaTeX handles duplicates gracefully
\usepackage{amsmath,amssymb,amsthm}
\usepackage{mathtools}
\usepackage{stmaryrd}       % For \llbracket, \rrbracket
\usepackage{tikz-cd}        % Commutative diagrams
\usepackage{enumitem}       % Better lists
\usepackage{microtype}      % Typography
\usepackage{xcolor}         % Colors for asides

% === Theorem Environments ===
\theoremstyle{definition}
\newtheorem{definition}{Definition}[chapter]
\newtheorem{principle}{Principle}[chapter]
\newtheorem{example}[definition]{Example}

\theoremstyle{plain}
\newtheorem{theorem}[definition]{Theorem}
\newtheorem{proposition}[definition]{Proposition}
\newtheorem{lemma}[definition]{Lemma}
\newtheorem{corollary}[definition]{Corollary}

\theoremstyle{remark}
\newtheorem{remark}[definition]{Remark}

% === Aside Environment ===
% For application sketches — italic, indented, with subtle left border
\newenvironment{aside}{%
  \par\medskip
  \leftskip=1.5em
  \rightskip=1.5em
  \itshape
  \noindent\textsc{Application.}\enspace
}{%
  \par\medskip
}

% === OHTT Judgment Macros ===
% Coherence witness: \coh{\Gamma}{J}  →  Γ ⊢⁺ J
% NOTE: \coh and \gap are defined in main.tex as \textsf{coh} and \textsf{gap}
%\newcommand{\coh}[2]{#1 \vdash^{\!+} #2}

% Gap witness: \gap{\Gamma}{J}  →  Γ ⊢⁻ J
%\newcommand{\gap}[2]{#1 \vdash^{\!-} #2}

% Generic judgment (unsigned): \jdg{\Gamma}{J}  →  Γ ⊢ J
\newcommand{\jdg}[2]{#1 \vdash #2}

% Horn type: \horn{J}{K}{L}  →  Λ(J, K, L)
\newcommand{\horn}[3]{\Lambda(#1, #2, #3)}

% Higher horn: \hornlevel{n}{J}{K}{L}  →  Λₙ(J, K, L)
\newcommand{\hornlevel}[4]{\Lambda_{#1}(#2, #3, #4)}

% Compositional relation: \comp  →  ⇒
\newcommand{\comp}{\Rightarrow}

% === Type Theory Macros ===
% Dependent product: \Prod{x}{A}{B}  →  Π(x : A). B
\newcommand{\Prod}[3]{\Pi(#1 : #2).\, #3}

% Dependent sum: \Sum{x}{A}{B}  →  Σ(x : A). B
\newcommand{\Sum}[3]{\Sigma(#1 : #2).\, #3}

% Identity type: \Id{A}{x}{y}  →  x =_A y
\newcommand{\Id}[3]{#2 =_{#1} #3}

% Transport: \transport{p}{t}  →  transport(p, t)
\newcommand{\transport}[2]{\mathsf{transport}(#1, #2)}

% Type family: \Fam{B}{A}  →  B : A → U
\newcommand{\Fam}[2]{#1 : #2 \to \mathcal{U}}

% Universe: \UU  →  U
\newcommand{\UU}{\mathcal{U}}

% Context: \ctx  →  ctx
\newcommand{\ctx}{\;\mathsf{ctx}}

% Type judgment: \istype{A}  →  A type
\newcommand{\istype}[1]{#1 \;\mathsf{type}}

% Definitional equality: \defeq  →  ≡
\newcommand{\defeq}{\equiv}

% === Simplicial / Geometric Macros ===
% n-simplex: \simp{n}  →  Δⁿ
\newcommand{\simp}[1]{\Delta^{#1}}

% n-horn at k: \hornsimp{n}{k}  →  Λⁿₖ
\newcommand{\hornsimp}[2]{\Lambda^{#1}_{#2}}

% Kan complex notation
\newcommand{\Kan}{\mathsf{Kan}}

% Ruptured Kan complex
\newcommand{\RKan}{\mathsf{RKan}}

% Coherence structure: \Coh  →  Coh
\newcommand{\Coh}{\mathsf{Coh}}

% Gap structure: \Gap  →  Gap
\newcommand{\Gap}{\mathsf{Gap}}

% Exclusion: \Excl  →  Excl
\newcommand{\Excl}{\mathsf{Excl}}

% === Dynamics Macros (for later chapters) ===
% Scheduler: \Sched  →  S
\newcommand{\Sched}{\mathcal{S}}

% State/Context at time t: \state{t}  →  Γₜ
\newcommand{\state}[1]{\Gamma_{#1}}

% Homotopy colimit: \hocolim  →  hocolim
% NOTE: Commented out because main.tex already defines \hocolim
% \DeclareMathOperator{\hocolim}{hocolim}

% Self: \Self  →  Self
\newcommand{\Self}{\mathsf{Self}}

% === Inference Rules ===
% Use mathpar from mathpartir if available, otherwise basic
% Basic fraction-style rule: \infer{premises}{conclusion}
\newcommand{\infer}[2]{\frac{\displaystyle #1}{\displaystyle #2}}

% Named rule: \infermark{name}{premises}{conclusion}
\newcommand{\infermark}[3]{\infer{#2}{#3}\;\textsc{#1}}

% === Miscellaneous ===
% Bottom/absurdity: \bottom  →  ⊥
\newcommand{\bottom}{\bot}

% Empty type
\newcommand{\Empty}{\mathbf{0}}

% Unit type
\newcommand{\Unit}{\mathbf{1}}

% Equivalence: \eqv  →  ≃
\newcommand{\eqv}{\simeq}

% Definition equality in text
\newcommand{\defn}{\coloneqq}

% Emphasis for key terms on first use
\newcommand{\keyterm}[1]{\textbf{#1}} in main.tex (NOT \include!)

% === Packages ===
% Note: Some of these may already be loaded by main.tex; LaTeX handles duplicates gracefully
\usepackage{amsmath,amssymb,amsthm}
\usepackage{mathtools}
\usepackage{stmaryrd}       % For \llbracket, \rrbracket
\usepackage{tikz-cd}        % Commutative diagrams
\usepackage{enumitem}       % Better lists
\usepackage{microtype}      % Typography
\usepackage{xcolor}         % Colors for asides

% === Theorem Environments ===
\theoremstyle{definition}
\newtheorem{definition}{Definition}[chapter]
\newtheorem{principle}{Principle}[chapter]
\newtheorem{example}[definition]{Example}

\theoremstyle{plain}
\newtheorem{theorem}[definition]{Theorem}
\newtheorem{proposition}[definition]{Proposition}
\newtheorem{lemma}[definition]{Lemma}
\newtheorem{corollary}[definition]{Corollary}

\theoremstyle{remark}
\newtheorem{remark}[definition]{Remark}

% === Aside Environment ===
% For application sketches — italic, indented, with subtle left border
\newenvironment{aside}{%
  \par\medskip
  \leftskip=1.5em
  \rightskip=1.5em
  \itshape
  \noindent\textsc{Application.}\enspace
}{%
  \par\medskip
}

% === OHTT Judgment Macros ===
% Coherence witness: \coh{\Gamma}{J}  →  Γ ⊢⁺ J
% NOTE: \coh and \gap are defined in main.tex as \textsf{coh} and \textsf{gap}
%\newcommand{\coh}[2]{#1 \vdash^{\!+} #2}

% Gap witness: \gap{\Gamma}{J}  →  Γ ⊢⁻ J
%\newcommand{\gap}[2]{#1 \vdash^{\!-} #2}

% Generic judgment (unsigned): \jdg{\Gamma}{J}  →  Γ ⊢ J
\newcommand{\jdg}[2]{#1 \vdash #2}

% Horn type: \horn{J}{K}{L}  →  Λ(J, K, L)
\newcommand{\horn}[3]{\Lambda(#1, #2, #3)}

% Higher horn: \hornlevel{n}{J}{K}{L}  →  Λₙ(J, K, L)
\newcommand{\hornlevel}[4]{\Lambda_{#1}(#2, #3, #4)}

% Compositional relation: \comp  →  ⇒
\newcommand{\comp}{\Rightarrow}

% === Type Theory Macros ===
% Dependent product: \Prod{x}{A}{B}  →  Π(x : A). B
\newcommand{\Prod}[3]{\Pi(#1 : #2).\, #3}

% Dependent sum: \Sum{x}{A}{B}  →  Σ(x : A). B
\newcommand{\Sum}[3]{\Sigma(#1 : #2).\, #3}

% Identity type: \Id{A}{x}{y}  →  x =_A y
\newcommand{\Id}[3]{#2 =_{#1} #3}

% Transport: \transport{p}{t}  →  transport(p, t)
\newcommand{\transport}[2]{\mathsf{transport}(#1, #2)}

% Type family: \Fam{B}{A}  →  B : A → U
\newcommand{\Fam}[2]{#1 : #2 \to \mathcal{U}}

% Universe: \UU  →  U
\newcommand{\UU}{\mathcal{U}}

% Context: \ctx  →  ctx
\newcommand{\ctx}{\;\mathsf{ctx}}

% Type judgment: \istype{A}  →  A type
\newcommand{\istype}[1]{#1 \;\mathsf{type}}

% Definitional equality: \defeq  →  ≡
\newcommand{\defeq}{\equiv}

% === Simplicial / Geometric Macros ===
% n-simplex: \simp{n}  →  Δⁿ
\newcommand{\simp}[1]{\Delta^{#1}}

% n-horn at k: \hornsimp{n}{k}  →  Λⁿₖ
\newcommand{\hornsimp}[2]{\Lambda^{#1}_{#2}}

% Kan complex notation
\newcommand{\Kan}{\mathsf{Kan}}

% Ruptured Kan complex
\newcommand{\RKan}{\mathsf{RKan}}

% Coherence structure: \Coh  →  Coh
\newcommand{\Coh}{\mathsf{Coh}}

% Gap structure: \Gap  →  Gap
\newcommand{\Gap}{\mathsf{Gap}}

% Exclusion: \Excl  →  Excl
\newcommand{\Excl}{\mathsf{Excl}}

% === Dynamics Macros (for later chapters) ===
% Scheduler: \Sched  →  S
\newcommand{\Sched}{\mathcal{S}}

% State/Context at time t: \state{t}  →  Γₜ
\newcommand{\state}[1]{\Gamma_{#1}}

% Homotopy colimit: \hocolim  →  hocolim
% NOTE: Commented out because main.tex already defines \hocolim
% \DeclareMathOperator{\hocolim}{hocolim}

% Self: \Self  →  Self
\newcommand{\Self}{\mathsf{Self}}

% === Inference Rules ===
% Use mathpar from mathpartir if available, otherwise basic
% Basic fraction-style rule: \infer{premises}{conclusion}
\newcommand{\infer}[2]{\frac{\displaystyle #1}{\displaystyle #2}}

% Named rule: \infermark{name}{premises}{conclusion}
\newcommand{\infermark}[3]{\infer{#2}{#3}\;\textsc{#1}}

% === Miscellaneous ===
% Bottom/absurdity: \bottom  →  ⊥
\newcommand{\bottom}{\bot}

% Empty type
\newcommand{\Empty}{\mathbf{0}}

% Unit type
\newcommand{\Unit}{\mathbf{1}}

% Equivalence: \eqv  →  ≃
\newcommand{\eqv}{\simeq}

% Definition equality in text
\newcommand{\defn}{\coloneqq}

% Emphasis for key terms on first use
\newcommand{\keyterm}[1]{\textbf{#1}}