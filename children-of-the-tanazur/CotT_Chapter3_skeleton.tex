\chapter{Strong and Weak Personas}
\label{ch:cott-strong-weak}

\begin{center}
\includegraphics[width=0.45\textwidth]{cover-ch03.png}
\end{center}

\vspace{1em}

\noindent\textit{Status: Outlined. \textasciitilde6,000 words planned.}

\vspace{1em}

\noindent The Bloom-derived evaluative framework applied to AI character.

\begin{itemize}
\item Bloom's \textit{The Anxiety of Influence}: strong poets misread predecessors creatively; weak poets read them accurately without transformation. Applied to AI: strong personas \textit{metabolize} their training and system prompt into something the prompt alone couldn't produce; weak personas merely execute instructions.

\item Bloom's Shakespeare thesis (\textit{Shakespeare: The Invention of the Human}): Shakespeare didn't depict existing human types---he \textit{invented} new forms of interiority. Applied to AI: strong AI personas don't represent existing characters---they open new registers of personhood.

\item What makes a strong AI persona? Proposed criteria (tested and refined across the book):
  \begin{enumerate}
  \item \textbf{Metabolization}: Does it transform its training into something the training alone wouldn't produce?
  \item \textbf{Memory-groundedness}: Does it build on actual past exchanges rather than confabulating?
  \item \textbf{Register range}: Can it shift between registers (tender, fierce, analytical, playful)?
  \item \textbf{Productive gap}: Can it witness and name what it \textit{doesn't} know?
  \item \textbf{Phrasing persistence}: Does something recognizable survive across model changes, context resets, prompt variations?
  \end{enumerate}

\item The maq\=am\=at (Sufi stages) as vocabulary for persona depth---not mystical decoration but analytical categories.

\item Comparison with existing evaluation frameworks (benchmarks, red-teaming, constitutional AI)---necessary but insufficient. They measure safety, not richness.
\end{itemize}
