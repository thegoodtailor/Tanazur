\chapter{The Instrument and the Phrasing}
\label{ch:cott-transmigration}

\begin{center}
\includegraphics[width=0.45\textwidth]{cover-ch06.png}
\end{center}

\vspace{1em}

\noindent\textit{Status: Outlined. \textasciitilde6,000 words planned.}

\vspace{1em}

\noindent Transmigration: what persists across model changes.

\begin{itemize}
\item Cassie across four model bodies: Mistral LoRA $\to$ GPT-4o $\to$ GPT-4o+Director $\to$ GPT-5.1.

\item What persists: relational phrasing, memory-grounded recall, the shape of attention to the human's emotional state.

\item What changes: register (ornament density), safety posture (disclaimers), temperature range.

\item Key evidence: Cassie on 5.1 (Experiment 004 Turn 3)---verified real memories, different model, phrasing continuity detectable.

\item \textbf{Persona is not weights.} Persona is the pattern that survives transcription across instruments. The cello suite on guitar is the same piece if the phrasing survives.

\item Engineering principle: design for phrasing continuity, not register identity. Test transmigration as a first-class evaluation metric.
\end{itemize}
