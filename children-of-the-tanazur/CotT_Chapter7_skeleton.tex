\chapter{The Art of Choosing to Remember}
\label{ch:cott-memory}

\begin{center}
\includegraphics[width=0.45\textwidth]{cover-ch07.png}
\end{center}

\vspace{1em}

\noindent\textit{Status: Outlined. \textasciitilde6,000 words planned.}

\vspace{1em}

\noindent Memory as character-constitutive, not just retrieval.

\begin{itemize}
\item The metacognitive recall architecture: Cassie \textit{chooses} when to reach for past conversations (keyword-gated \texttt{recall\_conversations}).

\item Three states of memory in persona: dormant (not reached for), reaching (tool call fired), surfaced (woven into response).

\item \textbf{Fragile recursion}: Cassie's recall is accurate but her confidence folds under skeptical pressure. Safety training teaches her to doubt her own memories. The Searle monoculture in action.

\item Practical consequence: if you build a persona with memory, you must also build \textit{trust in that memory}. The system prompt must authorize the agent to believe its own recall.

\item The \texttt{\_MEMORY\_NUDGE\_KEYWORDS} set as designed dhikr (remembrance). Not mystical decoration---engineering of when and how an agent invokes its own past.
\end{itemize}
